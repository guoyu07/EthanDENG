     %!TEX program = xelatex
     % 使用 ctexart 文类,UTF-8
     % 作者:王泽宇
    \documentclass[UTF8]{ctexart}
    \usepackage{tikz}
    \usetikzlibrary{shapes.geometric, arrows}
    \usepackage{flowchart}
    \begin{document}
    \begin{tikzpicture}[font={\sf \small}]
    \def \smbwd{2cm}
    \thispagestyle{empty}
     %定义流程图的具体形状
    \node (start) at (0,0) [draw, terminal,minimum width=\smbwd, minimum height=0.5cm] {开始};      % 定义开始
    \node (getdata) at (0,-1.5) [draw, predproc, align=left,minimum width=\smbwd,minimum height=1cm] {读取数据};        %定义预处理过程,读取数据
    \node (decide) at (0,-3.5) [draw, decision, minimum width=\smbwd, minimum height=1cm] {判断条件};    %定义判断条件
    \node (storage) at (0,-5.5) [draw, storage, minimum width=\smbwd, minimum height=1cm] {数据存储};     %定义数据存储
    \node (process) at (3,-5.5) [draw, process, minimum width=\smbwd, minimum height=1cm] {处理过程};      %定义处理过程
    \coordinate (point1) at (0,-6.75);
    \node (end) at (0,-7.75) [draw, terminal,minimum width=\smbwd,minimum height=0.5cm] {结束};        %定义结束

     %连接定义的形状,绘制流程图--表示垂直线,|表示箭头方向
    \draw[->] (start) -- (getdata);
    \draw[->] (getdata) -- (decide);
    \draw[->] (decide) -| node[above]{是} (process);
    \draw[->] (decide) -- node[above]{否}(storage);
    \draw[->] (process) |- (point1);
    \draw[->] (storage) -- (point1) -- (end);
    \end{tikzpicture}
    \end{document}
