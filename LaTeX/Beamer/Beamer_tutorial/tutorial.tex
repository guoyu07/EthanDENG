\documentclass[10pt]{beamer}

\usetheme[menuwidth={0.5\paperwidth}]{erlangen}
\setbeamercovered{transparent=20}
\setbeamerfont{frametitle}{series=\bfseries}
\usepackage[hyperref,UTF8,space]{ctex}
\hypersetup{colorlinks=false,
            colorlinks=black,
            pdfborder=100,
            citecolor=black}
\usepackage{mathpazo}
\usefonttheme{serif}
\setbeamerfont{block title}{series=\bfseries}
\usepackage{graphicx}

\renewcommand\contentsname{目录}

\begin{document}

\title[使用 Beamer 制作学术讲稿]{\bfseries 使用 Beamer 制作学术讲稿}
\subtitle{\small Victory won\rq{}t come to me unless I go to it.}
\author[Ethan D.]{\small Ethan D \\{\color{erlangenblue}\url{http://ddswhu.com/}}}
\date{\today}
% \institute{Fudan University\\
  % \vspace{-0.2em}{\small Elegant\LaTeX{} } }

\begin{frame}[plain]
\titlepage
\end{frame}


\begin{frame}{目录安排}
  \tableofcontents
\end{frame}


\section{缘何}
\begin{frame}{选择 Beamer}
\noindent\textbf{Beamer 有哪些好处}
  \begin{enumerate}
    \item 学术偏好,就是任性
    \item 和 \LaTeX{} 搭配,兼容 \LaTeX{} 命令
    \item 高度可读性与清晰的逻辑
    \item 输出为 PDF,跨平台显示无差异
  \end{enumerate}
\textbf{怎么获取?}
\begin{enumerate}
  \item 安装 CTeX Full 版本
  \item 安装 TeX Live 或者其他发行版
\end{enumerate}

\end{frame}



\begin{frame}[fragile]{}
\begin{alertblock}{一个简单的例子}
  \begin{verbatim}
  \documentclass{beamer}
  \usepackage[space,space,hyperref]{ctex}
  \usetheme{warsaw}
  \author{ddswhu}
  \title{First Start}
  \begin{document}
  \frame{\titlepage}
  \begin{frame}[c]\frametitle{frame title}
      The context goes here
  \end{frame}
  \end{document}
  \end{verbatim}
\end{alertblock}
\end{frame}

\section{文档结构}
\begin{frame}[c,fragile]\frametitle{帧(frame)}
我们看到的每一页 PDF 即是所谓幻灯片的帧,Beamer 的帧分为三类,帧对于 beamer 就像书页对于书。
\begin{enumerate}
  \item 标题页帧  \verb|\frame{\titlepage}|
  \item 普通帧
  \begin{verbatim}
  \begin{frame}[对齐方式]\frametitle{帧标题}
  内容
  \end{frame}
  \end{verbatim}
  \vspace*{-4ex}
  \item 空白帧
  \begin{verbatim}
  \begin{frame}[plain]
  空白帧没有标题
  \end{frame}
  \end{verbatim}
\end{enumerate}

\end{frame}

\begin{frame}[c,fragile]\frametitle{标题页帧}
标题页帧是我们的第一帧,在这上面,我们会有标题、作者、时间、机构,LOGO 等信息,可以显示在标题的主要信息如下:
\begin{exampleblock}{标题页信息}
\begin{verbatim}
  \title[short title]{long title}
  \subtitle[short subtitle]{long subtitle}
  \author[short name]{long name}
  \date[short date]{long date}
  \institution[short name]{long name}
  \titlegraphic{\includegraphics[width=0.17\textwidth]{ias.pdf}}
\end{verbatim}

\end{exampleblock}
\end{frame}


\begin{frame}[c,fragile]\frametitle{目录与节、小节}
\noindent\textbf{目录}
\begin{verbatim}
  \begin{frame}\frametitle{Outline}
  \tableofcontents[part=1,pausesections]
  \end{frame}
\end{verbatim}
\textbf{节与小节}
\begin{verbatim}
  \section{section name}
  \subsection{subsection name}
  \subsubsection{sub-subsection name}
  \section*{section name}
\end{verbatim}
最后一个命令生成一个节(section),但是,这个节不进入目录中。
\end{frame}


\section{列表环境}
\begin{frame}[c,fragile]\frametitle{列表环境--itemize}
三类列表环境,包括无序列表(itemize)、有序列表(enumerate)、描述列表(description)。其中前两者非常常用,使用非常简单。\\
\noindent\textbf{无序列表 itemize 示例}
\begin{columns}[c]
\begin{column}{0.55\textwidth}
  \begin{verbatim}
  \begin{itemize}
    \item The first item
    \item The second item
    \item The third item
    \item The fourth item
  \end{itemize}
  \end{verbatim}
\end{column}
\begin{column}{0.45\textwidth}
\begin{itemize}
  \item The first item
  \item The second item
  \item The third item
  \item The fourth item
\end{itemize}
\end{column}
\end{columns}
\end{frame}

\begin{frame}[fragile]{列表环境--enumerate}
\noindent\textbf{有序列表 enumerate 示例}
\begin{columns}[c]
\begin{column}{0.55\textwidth}
  \begin{verbatim}
  \begin{enumerate}
    \item The first item
    \item The second item
    \item The third item
    \item The fourth item
  \end{enumerate}
  \end{verbatim}
\end{column}
\begin{column}{0.42\textwidth}
\begin{enumerate}
  \item The first item
  \item The second item
  \item The third item
  \item The fourth item
\end{enumerate}
\end{column}
\end{columns}

\end{frame}

\begin{frame}[fragile]{列表环境--description}
\noindent\textbf{描述列表 description 示例}
  \begin{verbatim}
  \begin{description}
    \item[First Item] Description of first item
    \item[Second Item] Description of second item
    \item[Third Item] Description of third item
    \item[Forth Item] Description of forth item
  \end{description}
  \end{verbatim}
  \vspace*{-5ex}
  \begin{description}
    \item[First Item] Description of first item
    \item[Second Item] Description of second item
    \item[Third Item] Description of third item
    \item[Forth Item] Description of forth item
  \end{description}
\end{frame}

\section{段落文本}
\begin{frame}[fragile]{文本命令}
\begin{columns}
  \begin{column}{0.6\paperwidth}
  \begin{verbatim}
      \emph{Sample Text}
      \textbf{Sample Text}
      \textit{Sample Text}
      \textsl{Sample Text}
      \alert{Sample Text}
      \textrm{Sample Text}
      \textsf{Sample Text}
      \textcolor{green}{Sample Text}
      \structure{Sample Text}
  \end{verbatim}
  \end{column}
  \begin{column}{0.39\paperwidth}
    \emph{Sample Text}\\
    \textbf{Sample Text}\\
    \textit{Sample Text}\\
    \textsl{Sample Text}\\
    \alert{Sample Text}\\
    \textrm{Sample Text}\\
    \textsf{Sample Text}\\
    \textcolor{green}{Sample Text}\\
    \structure{Sample Text}\\
  \end{column}
\end{columns}
\end{frame}

\begin{frame}[fragile]{字体与字号}
\noindent\textbf{字体主题:}\verb|\usefonttheme[onlymath]{serif}|\\
\noindent\textbf{字体大小:}\verb|\documentclass[11pt]{beamer}|,可选项为 10-11-12pt\\
\noindent\textbf{字体族\phantom{大}:}\verb|\usepackage{helvet}|,可选项如下:
\begin{exampleblock}{字体宏包}
\begin{verbatim}
  serif   avant    bookman   chancery  charter
  euler   helvet   mathtime  mathptm   mathptmx
  newcent palatino pifont    utopia
\end{verbatim}
\end{exampleblock}
\end{frame}


\section{帧内结构}

\begin{frame}[c,fragile]\frametitle{分栏}
\noindent\textbf{多栏显示:}在我们做讲稿的时候,有时候为了使得幻灯片更加工整,充实,会在一旁插入图片、表格或者说明性的文字,这个时候可以使用 beamer 中的多栏环境(columns)或者 \LaTeX{} 中的子页环境(minipage)。
\begin{verbatim}
  \begin{columns}
  \column{.xx\textwidth}
    First column text and/or code
  \column{.xx\textwidth}
    Second column text and/or code
  \end{columns}
\end{verbatim}
\end{frame}

\begin{frame}[c]\frametitle{区块}
Beamer 中一个非常有特色的是它带了一些区块(block),这些区块环境与普通文本能很好地区分开来,适用于各种定理,引理以及示例。默认的 Beamer 有如下环境:
\begin{columns}
\column{0.4\textwidth}
\begin{exampleblock}{Beamer 自带区块环境}
\begin{description}
\item[block] 普通环境
\item[theorem] 定理环境
\item[lemma] 引理环境
\item[proof] 证明环境
\item[corollary] 推论环境
\item[example] 示例环境
\item[alertblock] 警示环境
\end{description}
\end{exampleblock}
\column{0.6\textwidth}
\begin{alertblock}{勾股定理}
我们熟知的勾股定理
\begin{equation}
a^{2}+b^{2}=c^{2}
\end{equation}
\end{alertblock}
\end{columns}

\end{frame}

\section{图与表格}

\begin{frame}[c,fragile]\frametitle{插入表格}
\noindent插图与插表格的命令和 \LaTeX{} 没有区别,插表的命令如下
\begin{verbatim}
  \begin{table}[tb]
  \centering
  \caption{Caption here\label{tab:tablename}}
    \begin{tabular}{l|cc}  \hline
    \textbf{column 1} & \textbf{column 2} & \textbf{column 3} \\ \hline
    Hello  & Beamer & NAN \\ \hline
    $\alpha+\beta$ & $\gamma+\eta$ & 34\% \\ \hline
    \end{tabular}
  \end{table}
\end{verbatim}
\end{frame}

\begin{frame}[c,fragile]\frametitle{插入图片}
\noindent 插图与插表格的命令和 \LaTeX{} 没有区别,插图的命令如下
\begin{verbatim}
\begin{figure}[tb]
  \centering
  \includegraphics[width=0.9\textwidth]{figure.png}
  \caption{Caption here\label{fig:figure1}}
\end{figure}
\end{verbatim}
\end{frame}



\begin{frame}[c,fragile]\frametitle{图表效果}
\begin{columns}[c]
\column{0.5\textwidth}
  \begin{table}[tb]
  \centering
  \caption{Caption here\label{tab:tablename}}
    \begin{tabular}{l|cc}  \hline
    \textbf{column 1} & \textbf{column 2} & \textbf{column 3} \\ \hline
               Hello  & Beamer            & NAN               \\ \hline
    $\alpha+\beta$    & $\gamma+\eta$     & 34\%              \\ \hline
    \end{tabular}
  \end{table}
\column{0.5\textwidth}
\begin{figure}[tb]
  \centering
  \includegraphics[width=0.75\textwidth]{figure.png}
  \caption{Caption here\label{fig:figure1}}
\end{figure}
\end{columns}

\end{frame}
\section{主题}
\begin{frame}[c,fragile]\frametitle{主题}
Beamer 因为有各种主题才使得有生气,不过,大部分人还在用默认的主题,其中,Warsaw 用的非常多,建议避免。默认的主题如下:
\begin{exampleblock}{Beamer 自带主题}
\begin{verbatim}
Antibes     Bergen      Berkeley     Berlin
Boadilla    Copenhagen  Darmstadt    Dresden
Frankfurt   Goettingen  Hannover     Ilmenau
Juanlespins Madrid      Malmoe       Marburg
Montpellier Paloalto    Pittsburgh   Rochester
Singapore
\end{verbatim}
\end{exampleblock}
可以查看 {\color{erlangenblue}\url{http://www.ctan.org/tex-archive/macros/latex/contrib/beamer/doc/}} 上各种主题的 PDF 效果。
\end{frame}

\begin{frame}[c,fragile]\frametitle{颜色主题}
颜色主题有三种,分别是 colortheme,inner color theme,outter color theme。 colortheme 控制着全部的颜色,inner color theme 控制着帧内一些元素的颜色,主要是 block,outter color theme 控制着 headline,footline,sidebar 等元素的颜色。使用的方法都是一致的。
\begin{verbatim}
\usecolortheme{default}
\end{verbatim}
\begin{exampleblock}{Beamer 中的颜色主题}
\begin{description}
  \item[color theme] albatross crane beetle dove fly seagull wolverine beaver
  \item[inner color] lily orchid rose
  \item[outter color] whale seahorse dolphin
\end{description}
\end{exampleblock}

\end{frame}

\begin{frame}[c]\frametitle{了解更多}
\noindent\textbf{想了解更多?}\\
\begin{itemize}
    \item The Beamer Class User Guide for v3.10(官方文档)
    \item Beamer v3.0 Guide(Ki-JooKim)
    \item Beamer v3.0 指南 (黄旭华译)
    \item A Beamer Tutorial in Beamer(本文内容来源)
\end{itemize}

\noindent 本幻灯片主题来源于 \LaTeX{}Studio,传送门:{\color{erlangenblue}\url{http://www.latexstudio.net/fau-erlangen-beamer-template/}}

\end{frame}


\begin{frame}[c]\frametitle{广告时间}
Elegant\LaTeX{} 项目组致力于设计美观的 \LaTeX{} 系列模板,我们的主页是 {\color{erlangenblue}\url{http://elegantlatex.org/}}!欢迎关注我们的微博:@ElegantLaTeX,Elegant\LaTeX{} 目前成员有:
\begin{itemize}
\item Ethan D ({\color{erlangenblue}\url{http://ddswhu.com/}})
\item 勤劳的小L ({\color{erlangenblue}\url{http://liam0205.me/}})
\item 逐鹿人 ({\color{erlangenblue}\url{http://wangmurong.org.cn/}})
\item fitsir ({\color{erlangenblue}\url{http://fitsir.me/}})
\item WD (\texttt{Error 404 -- Not Found})
\end{itemize}

如果你有一颗乐于奉献的心、对 \LaTeX{} 比较感兴趣,或者对设计有兴趣、独特见解,欢迎加入我们!

\end{frame}

\end{document}
