%!TEX root = macro.tex
\chapter{Search and Unemployment Insurance}
In a equilibrium setup: time period $t=0,1,2,\ldots$, one consumption good. A continum of agents
\begin{equation}
\mathbb{E}_{0}\sum_{t=0}^{\infty} \beta ^{t} [u(c_{t})-\phi(a_{t})]
\end{equation}
where $\beta \in (0,1)$, $u^{\prime} > 0$, $u^{\prime\prime}<0$. And $\phi^{\prime} > 0,\phi^{\prime\prime}>0$ . $a_{t}\in [0,+\infty)$ is the agent\rq s effort.

Search:
\begin{itemize}
    \item Any agent, when unemployed, making effort $a_{t}$ in period $t$, finds a job with prob $\pi(a_{t})\in [0,1],\forall\,a_{t}$. Assume $\pi^{\prime} > 0, \pi^{\prime\prime}<0$.
    \item Unemployed agent didn't generate income.
    \item Jobs are identical, all pay a constant $y=1$ units of the good in each period.
    \item Once employed, the agent has in each period an exogenous prob $\delta\in (0,1)$ to be terminated in which case he goes back to labor mkt, unemployed.
    \item There is a government in the model who runs an umemployment insurance policy which has two dimensions:
    \begin{itemize}
        \item $b$: unemployment insurance benefits paid per period to any unemployed worker
        \item $\tau$: income tax per period on each employed worker
    \end{itemize}
\end{itemize}

Question: what is the optimal $(b,\tau)$?

Consider an symmetric steady-state(stationary) equlibrium of the model in which
\begin{enumerate}
    \item stationary: all ``equilibrium objects'' are time invariant.
    \item symmetric: all unemployed workers take the same action in search.
\end{enumerate}

Describing a-symmetric stationary equilibrium taking $b$ and $\tau$ as given.

Defn: A S-S equilibrium of the model is a vector $\big\{ a^{*},E,U\big\}$, where
\begin{align*}
    a^{*} &= \text{equilibrium search effort of the unemployed}\\
    E     &= \text{measure of employed workers at the begining (end) of the peroid}\\
    U     &= \text{measure of the unemployed workers \ldots}, E+U=1
\end{align*}
such that
\begin{enumerate}
\item $a^{*}$ solves the unemployment worker\rq s problem
\begin{align}
 V_u   &= \max_{a\in [0,\infty)} \Big\{\pi(a)\big[u(y-\tau)+\beta[\delta V_u+(1-\delta)V_e]\big]+(1-\pi(a))\big[u(b)+\beta V_u\big]-\phi(a)\Big\}\\
 V_{e} &= u(y-\tau) +\beta\big\{\delta V_{u}+(1-\delta)V_{e}\big\}
\end{align}
\item $E_{t}=\text{const.}$ in equlibrium.
\begin{gather}
E_{t+1} = E_{t}(1-\delta)+(1-E_{t}) \pi(a^{*}) = E_{t}.\\
E(1-\delta)+(1-E)\pi(a^{*}) = E.\label{flow out and in}\\
\delta E = \pi(a^{*}) (1-E). \label{flow out and in}
\end{gather}
The left side of \cref{flow out and in} is flow out of employment, and the right side is flow into the employment.
\end{enumerate}

Rewrite as:
\begin{align}
V_{u} &= \max_{a\in [0,\infty)} \Big\{\pi(a) V_{e} + (1-\pi(a))\big[u(b)+\beta V_{u}\big]\Big\}\label{prime1}\\
\big\(1-\beta(1-\delta)\big\) V_{e} &= u(y-\tau) +\beta \delta V_{u}\label{prime2}\\
\delta E &= \pi(a^{*}) (1-E)\label{prime3}
\end{align}

FOC for \cref{prime1}
\begin{align}
\pi^{\prime}(a)\big[V_{e}-\beta V_{u}-u(b)\big] &= \phi^{\prime}(a)\label{4}\\
V_{e}-\beta V_{u} - u(b) &= \frac{\phi^{\prime}(a)}{\pi^{\prime}(a)}\label{prime4}
\end{align}

Plug \cref{prime4} into \cref{prime1}
\begin{align}
V_{u} & = \pi(a^{*})\big[V_{e}-\beta V_{u}-u(b)\big]+u(b)+\beta V_{u} -\phi(a^{*})\\
      & = \pi(a^{*})\frac{\phi^{\prime}(a^{*})}{\pi^{\prime}(a^{*})}+u(b)+\beta V_{u}-\phi(a^{*})
\end{align}

Then we have
\begin{equation}
    (1-\beta) V_{u}-u(b)=\pi(a^{*})\frac{\phi^{\prime}(a^{*})}{\pi^{\prime}(a^{*})}-\phi(a^{*})
\end{equation}
