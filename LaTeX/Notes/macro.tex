%!TEX program = xelatex
\RequirePackage[l2tabu, orthodox]{nag}
\documentclass{book}
% 9 LaTeX packages everyone should use
\usepackage{amsmath}
\usepackage[a4paper]{geometry}
\usepackage{graphicx}
\usepackage{microtype}
\usepackage{siunitx}
\usepackage{booktabs}
\usepackage[colorlinks=false, pdfborder={0 0 0}]{hyperref}
\usepackage{cleveref}

\usepackage[scaled=0.92]{helvet}    % set Helvetica as the sans-serif font
\renewcommand{\rmdefault}{ptm}      % set Times as the default text font
% The following loads mtpro and defines some common MTPro options [2, 4]
\usepackage[subscriptcorrection,slantedGreek,nofontinfo]{mtpro2}

\usepackage{xcolor}
% \definecolor{myblue}{RGB}{100,0,239}
\everymath\expandafter{\the\everymath \color{magenta}}
\everydisplay\expandafter{\the\everydisplay \color{magenta}}

\usepackage[Lenny]{fncychap}


\begin{document}

\chapter{A Primer on Continuous-time Economic Dynamics}

\section{Linear Differential Equation Systems} % (fold)
\label{sec:linear_differential_equation_systems}

\subsection{Simplest case} % (fold)
\label{sub:simplest_case}

We begin with the simple linear first-order differential equation,

\begin{equation*}
\dot{x}=ax, \quad x(0)=x_{0}
\end{equation*}

The general solution is
\begin{equation*}
x(t)=c_{0}e^{at},
\end{equation*}
and the initial condition is satisfied by the particular solution,
\begin{equation*}
x(t)=x_{0}e^{at}
\end{equation*}
The growth rate ofxis given by $a$.

Next, we solve the two-dimensional system given by
\begin{equation*}
    \dot{x}=ax,\quad x(0)=x_{0}
\end{equation*}
and
\begin{equation*}
    \dot{y}=by,\quad y(0)=y_{0}.
\end{equation*}
It is useful to write these in matrix form as
\begin{equation*}
\begin{bmatrix}
\dot{x}\\ \dot{y}
\end{bmatrix}
=\begin{bmatrix}
a & 0 \\
0 & b
\end{bmatrix}
\begin{bmatrix}
x\\
y
\end{bmatrix}
,\quad
\begin{bmatrix}
x(0)\\
y(0)
\end{bmatrix}
=\begin{bmatrix}
x_{0}\\y_{0}
\end{bmatrix}.
\end{equation*}

The solution for this two-dimensional system is simply
\begin{equation*}
\begin{bmatrix}
x(t)\\y(t)
\end{bmatrix}
=\begin{bmatrix}
x_{0}e^{at}\\y_{0}e^{bt}
\end{bmatrix}.
\end{equation*}
Each of these problems, the one-dimensional and the two-dimensional, are examples of initialvalue problems. This system possesses a single steady state, $x_{0}=0$ and $y_{0}=0$. A phase diagram plots out $y(t)$ as a function of $x(t)$ for different possible initial values of $x$ and $y$. The resulting locus relating $y$ to $x$ at each time given any particular initial values, $x_{0}$ and $y_{0}$, depicts a trajectory along which $x$ and $y$ move over time. Arrows are usually drawn to depict the direction of motion in the $xy-$plane as $t$ increases.

Consider three cases. In the first, the parameters $a$ and $b$ are both positive. For any initial point, $(x_{0},y_{0})$,  that is not the steady state, a trajectory moves away from the origin. This is the unstable case. In the second, $a$ and $b$ are both negative and any trajectory converges to the steady state. This is the stable case. In this case, trajectories asymptotically approach the steady state parallel to the $y-$axis if $b<a<0$, and conversely. In the third case, one root, $a$ or $b$ is positive and the other is negative. For the example in which $a>0>b$, any trajectory such that $x_{0}=0$ converges to the steady state at the rate $b$. Any trajectory such that $x_{0}\not =0$ moves away from the steady state and converges towards the $x-$axis asymptotically. This is the saddle-path stable case.

Consider an example of saddle-path dynamics such that $a=1$ and $b=−1$. The solution is given by

\begin{equation*}
\begin{bmatrix}
x(t)\\
y(t)
\end{bmatrix}
= \begin{bmatrix}
x_{0}e^{t}\\
y_{0}e^{-t}
\end{bmatrix}
\end{equation*}
As $t$ grows without bound, $y(t)$ approaches zero as $x(t)$ grows without bound as $x_{0}e^{t}$.
% subsection simplest_case (end)

\subsection{General case} % (fold)
\label{sub:general_case}

Let $z$ be an $n-\text{dimensional}$ column vector over the reals and $\dot{z}=Az$ where $z(0) = z_{0}$. The matrix, $A$, is $n\times n$ with constant real coefficients. Impose the restriction that $A$ is nonsingular.

For example, consider the two-dimensional differential equation system,
\begin{equation*}
\begin{bmatrix}
\dot{x}\\
\dot{y}
\end{bmatrix}
= \begin{bmatrix}
a & c\\
d & b
\end{bmatrix}
\begin{bmatrix}
x\\
y
\end{bmatrix},
\quad
\begin{bmatrix}
x(0)\\
y(0)
\end{bmatrix}
= \begin{bmatrix}
x_{0}\\
y_{0}
\end{bmatrix}.
\end{equation*}

We handle this problem by changing coordinates. We want to find a change of basis from
$\begin{bmatrix}
x\\y
\end{bmatrix}$
to a new basis,
$\begin{bmatrix}
\hat{x}\\
\hat{y}
\end{bmatrix}$
such that the initial-value problem becomes

\begin{equation*}
\frac{d}{dt}
\begin{bmatrix}
\dot{x}\\
\dot{y}
\end{bmatrix}
= \begin{bmatrix}
\hat{a} & 0\\
0 & \hat{b}
\end{bmatrix}
\begin{bmatrix}
\hat{x}\\
\hat{y}
\end{bmatrix},
\quad
\begin{bmatrix}
\hat{x}(0)\\
\hat{y}(0)
\end{bmatrix}
= \begin{bmatrix}
\hat{x}_{0}\\
\hat{y}_{0}
\end{bmatrix}.
\end{equation*}
which has the solution given in section \ref{sub:simplest_case},
\begin{equation*}
\begin{bmatrix}
\hat{x}\\
\hat{y}
\end{bmatrix}
= \begin{bmatrix}
\hat{x}_{0}e^{\hat{a}t}\\
\hat{y}_{0}e^{\hat{b}t}
\end{bmatrix}
\end{equation*}
This is done by finding a nonsingular $2\times 2$ matix, $M$, with constant coefficients such that
\begin{equation*}
M^{-1}\begin{bmatrix}
a & c\\
d & b
\end{bmatrix} M
= \begin{bmatrix}
\hat{a} & 0\\
0 & \hat{b}
\end{bmatrix}
\end{equation*}
and
\begin{equation*}
\begin{bmatrix}
x\\
y
\end{bmatrix}
= M \begin{bmatrix}
\hat{x}\\
\hat{y}
\end{bmatrix}.
\end{equation*}
We can always do this given the assumptions madeso far. The differential equation systems are
identical but expressed in different coordinates.
% subsection general_case (end)

% section linear_differential_equation_systems (end)

\end{document}
