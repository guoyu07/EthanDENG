%!TEX program = xelatex

\documentclass{article}
\usepackage{xcolor}
\definecolor{god}{RGB}{184,134,11} %184,134,11
\usepackage{hyperref}
\usepackage{graphicx}

\usepackage{listings,relsize}
\lstset{% general command to set parameter(s)
language=R,
basicstyle=\sffamily, % print whole listing small
keywordstyle=\color{god},
commentstyle=\color{gray}, % white comments
stringstyle=\sffamily, % typewriter type for strings
showstringspaces=false,
numbers=left, numberstyle=\tiny, stepnumber=1, numbersep=5pt, %
frame=shadowbox,
rulesepcolor=\color{gray},
columns=fullflexible,
morekeywords={pkgtest,url,index,filepre,urlpre,transform,ndigits,filetype,downloader,tdownloader,TRUE,FALSE}
} %



\usepackage{geometry}
\geometry{
            a4paper,
            left=27mm,  %% or inner=23mm
            right=27mm, %% or outer=18mm
            top=25.4mm, bottom=25.4mm,
            headheight=2.17cm,
            headsep=4mm,
            footskip=12mm
}

%%字体设置

\usepackage[no-math,cm-default]{fontspec}
\defaultfontfeatures{Mapping=tex-text}
\RequirePackage{xunicode}
\RequirePackage{xltxtra}
\setmainfont[Ligatures=TeX]{Minion Pro} %  (\textrm)
\setsansfont{Myriad Pro} %  (\textsf)
\setmonofont{Monaco}%Palatino Linotype
%-中文字体设置-%
\usepackage{xeCJK}
\setCJKmainfont[BoldFont={方正黑体简体},ItalicFont={方正楷体简体}]{方正书宋简体}
%方正书宋_GBK Adobe Song Std L华文中宋
\setCJKsansfont[BoldFont={方正黑体简体}]{方正中等线简体}
\setCJKmonofont{微软雅黑Monaco}
\XeTeXlinebreaklocale "zh"
\XeTeXlinebreakskip = 0pt plus 1pt
\newfontfamily\gara{Adobe Garamond Pro}
\linespread{1.3}
\title{\bfseries{\color{blue}Download}{\color{magenta}Helper} 用户手册}
\author{\Large\gara\itshape{{\color{blue}E}than {\color{magenta}D}eng} \\ Version 2.0}
\begin{document}
\maketitle
\section{功能介绍}

本 R 文件(\href{http://stack.ddswhu.com/R/download/downloadhelper.R}{DownloadHelper.R} 简称 DH)主要功能是批量下载相近文件名或者有文件名列表的文件。原则上 DH 支持所有的文件格式,经过测试的文件类型有:\textsf{PDF} 文件, \textsf{Excel} 文件,\textsf{txt} 纯文本文件 , \textsf{LaTeX} 文件,\textsf{R} 文件, \textsf{Matlab} 文件,\textsf{MP3} 文件。DH 定义了四个函数,用户只需要使用最外层的函数 downloader 或者 tdownloader 即可批量下载相似地址的文件。

\section{函数说明}

DH 定义了如下四个函数:

\begin{itemize}
    \item pkgtest:检测是否已经安装某个宏包,参数为宏包名(参数:字符常量 x)
    \item transform:数值索引转换为统一位数格式(参数1:index;参数2:ndigits)
    \item downloader:下载文件函数 1.0 版本,参数及示例见后文。
    \item tdownloader:下载文件函数 2.0 版本,参数及示例见后文。
\end{itemize}

\subsection{pkgtest 函数说明}

其中 \textsf{pkgtest} 函数用于检测宏包是否安装,如果没有,则进行安装。由于 DH 需要借助 \textsf{downloader} 宏包,所以在使用 \textsf{downloader} 之前需要检测宏包是否安装,用户不需要自行检查。\textsf{downloader} 函数会自动调用 \textsf{pkgtest} 函数。\textsf{pkgtest} 使用示例如下

\begin{lstlisting}
pkgtest("xtable")
# output as follows
[1] "The required packages have been installed"
\end{lstlisting}

\subsection{transform 函数说明}

有时候我们可能会碰到如下形式的文件下载地址 \lstinline{~/chap002.pdf},此地址文件索引包含的是 \lstinline{002} 而不是 \lstinline{2}。考虑到这种情况,我们定义了 \textsf{transform} 函数,用以把普通数字转换成统一位数(不够位数则补零)的数字格式,也即 \lstinline{2} 转化为 \lstinline{002}。这个函数有两个参数,分别为 \lstinline{index} 和 \lstinline{ndigits},第一个参数表示需要转换的数字,也可以是向量。第二个参数是转换后数字的位数(也即转换后数字的长度,比如 002 位数为 3)。示例如下
\begin{lstlisting}
index <- 1:24
transform(index,ndigits=3)
# output as follows
 [1] "001" "002" "003" "004" "005" "006" "007" "008" "009" "010" "011" "012"
[13] "013" "014" "015" "016" "017" "018" "019" "020" "021" "022" "023" "024"
\end{lstlisting}

\subsection{downloader 函数说明}
\begin{itemize}
    \item urlpre:字符参数,下载文件网址不含文件名部分,必选参数;
    \item index: 文件名索引,不含拓展名,必选参数,可以为字符向量或者数值向量;
    \item transform:逻辑参数,补零转换,默认 \lstinline{transform = FALSE};
    \item ndigits:数值参数,总位数,比如需要得到 012,则 \lstinline{ndigits=3},默认 \lstinline{ndigits = 2};
    \item filepre:字符参数,文件名前缀,比如 chap,默认为空;
    \item filetype:字符参数,文件类型,文件拓展名,默认 \lstinline{filetype="pdf"}。
\end{itemize}

我们用一个例子来解释上面的参数。比如,现在要下载 \href{http://www.math.upenn.edu/~guffin/teaching/fall10/}{http://www.math.upenn.edu/~guffin/teaching/fall10/} 下 lectures 的课件,课件的地址为 \href{http://www.math.upenn.edu/~guffin/teaching/fall10/lectures/lecture-01.pdf}{http://www.math.upenn.edu/~guffin/teaching/fall10/lectures/lecture-01.pdf},我们对这几个参数一一解释。

\lstinline{urlpre} 是指下载文件地址中不含文件名部分,大部分情况下,都是最后一个斜杠的之前的内容(含斜杠)。在我们的这个例子中,\lstinline{urlpre = "http://www.math.upenn.edu/~guffin/teaching/fall10/lectures/"},

\lstinline{index} 是指下载文件的文件名的索引,比如这里文件名分别为 01.pdf--32.pdf,则 \lstinline{index = 1:32}。当然,可能有时候并不是顺序,则可以使用向量表示。\lstinline{index} 可以为字符向量或者数值向量。字符向量比如 \lstinline{index = c("hello","rintro","fname-esim")},也即文件名的一个列表(不含拓展名)。

\lstinline{transform} 是个逻辑值参数,可选项有 \lstinline{TRUE} 和 \lstinline{FALSE},前者表示进行数字格式转换,比如此例中,由于下载地址中是 \lstinline{01.pdf},需要将 \lstinline{1} 转换成 \lstinline{01} 这种形式,所以这个例子中 \lstinline{transform = TRUE}。

\lstinline{ndigits} 表示需要转换的位数,当然,只在 \lstinline{transform = TRUE} 时候才会生效。在我们这个例子中,由于索引转换之后为 \lstinline{01},所以位数为 2,\lstinline{ndigits = 2}。默认地,

\lstinline{filepre} 表示文件名前缀,字符参数,在我们这个例子中,文件名前缀参数为 \lstinline{filepre = "lecture-"}。

\lstinline{filetype} 字符参数,表示文件类型,准确来说应该是文件名后缀,在本例中,\lstinline{filetype = "pdf"},注意:必须要用双引号。

综合起来,完整的例子如下
\begin{lstlisting}
source("downloadhelper.R")
urlpre <- "http://www.math.upenn.edu/~guffin/teaching/fall10/lectures/"
index <- 2:4
downloader(urlpre,index,ndigits=2,transform=TRUE,filepre="lecture-",filetype="pdf")
\end{lstlisting}

\subsection{tdownloader 函数说明}
tdownloader 函数是 downloader 的升级版,本质上 tdownloader 使用 downloader 函数。使用 tdownloader 下载文件需要使用到两个参数,分别是 \lstinline{url} 和 \lstinline{index}。
\begin{itemize}
    \item url:文件下载示例网址,具有代表性的网址。
    \item index:文件名索引,同 downloader 函数参数。
\end{itemize}

tdownloader 目前仅支持三类地址下载,见使用示例。

\section{使用示例}
\subsection{downloader 下载示例}
\noindent\textbf{PDF 文件下载}\\
相关网页:\href{http://www.math.upenn.edu/~guffin/teaching/fall10/}{http://www.math.upenn.edu/~guffin/teaching/fall10/}
\begin{lstlisting}
source("downloadhelper.R")
urlpre <- "http://www.math.upenn.edu/~guffin/teaching/fall10/lectures/"
index <- 2:4
downloader(urlpre,index,ndigits=2,transform=TRUE,filepre="lecture-",filetype="pdf")
\end{lstlisting}

\noindent\textbf{R 文件下载}\\
相关网页:\href{http://www.rni.helsinki.fi/~pek/r-koulutus/index.html}{http://www.rni.helsinki.fi/~pek/r-koulutus/index.html}

\begin{lstlisting}
source("downloadhelper.R")
urlpre <- "http://www.rni.helsinki.fi/~pek/r-koulutus/"
index <- c("hello","rintro","fname-esim")
downloader(urlpre,index,filetype="R")
\end{lstlisting}

\noindent\textbf{txt 文件下载}\\
相关网页:\href{http://www1.umn.edu/statsoft/doc/statnotes/}{http://www1.umn.edu/statsoft/doc/statnotes/}

\begin{lstlisting}
source("downloadhelper.R")
urlpre <- "http://www1.umn.edu/statsoft/doc/statnotes/"
index <- 1:8
downloader(urlpre,index,transform=TRUE,ndigits=2,filepre="stat",filetype="txt")
\end{lstlisting}

\noindent\textbf{\LaTeX{} 文件下载}\\
相关网页: \href{https://gitorious.org/pkuthss/pkuthss/source/1bb0c28e36e7ddb7d6db0009a1d154e1e3ef4c6c:doc/chap}{https://gitorious.org/pkuthss/pkuthss/}

\begin{lstlisting}
source("downloadhelper.R")
urlpre <- "https://gitorious.org/pkuthss/pkuthss/raw/1bb0c28e36e7ddb7d6db000
          9a1d154e1e3ef4c6c:doc/chap/"
index <- 1:3
downloader(urlpre,index,filetype="tex",filepre="chap")
\end{lstlisting}

\noindent\textbf{Excel 表格下载}\\
相关网页:\href{http://www.pbc.gov.cn/publish/html/kuangjia.htm?id=2012s07.htm}{http://www.pbc.gov.cn/publish/html/}
\begin{lstlisting}
source("downloadhelper.R")
urlpre <- "http://www.pbc.gov.cn/publish/html/"
index <- 5:8
downloader(urlpre,index,transform=TRUE,filetype="xls",filepre="2012s")
\end{lstlisting}

\noindent\textbf{Matlab m 文件下载}\\
相关网页:\href{http://www.rni.helsinki.fi/~pek/software.html}{http://www.rni.helsinki.fi/~pek/software.html}

\begin{lstlisting}
source("downloadhelper.R")
urlpre <- "http://www.rni.helsinki.fi/~pek/software/smoothing/"
index <- c("llrev","llrcvev","llrssev","llrrev","llrrcvev","llrrssev","knnev","knncvev","knnssev","demo")
downloader(urlpre,index,filetype="m")
\end{lstlisting}


\noindent \textbf{MP3 文件下载}\\
相关网页:\href{http://download.dogwood.com.cn/online/grechjx/index.html}{http://download.dogwood.com.cn/online/grechjx/index.html}

\begin{lstlisting}
source("downloadhelper.R")
urlpre <- "http://download.dogwood.com.cn/online/grechjx/"
urlindex <- 4:6
downloader(urlpre,index,ndigits=2,transform=TRUE,filepre="WordList",filetype="mp3")
\end{lstlisting}


\subsection{tdownloader 下载示例}

\noindent \textbf{索引类型 1:常规字符文件列表}
\begin{lstlisting}
source("downloadhelper.R")
url <- "http://www.rni.helsinki.fi/~pek/software/smoothing/llrev.m"
index <- c("llrev","llrcvev","llrssev","llrrev","llrrcvev","llrrssev","knnev","knncvev","knnssev","demo")
tdownloader(url,index)
\end{lstlisting}

\noindent \textbf{索引类型 2:含字符+常规数字文件}
\begin{lstlisting}
url <- "https://gitorious.org/pkuthss/pkuthss/source/1bb0c28e36e7ddb7d6db0009a1d154e1e3ef4c6c
       :doc/chap/chap1.tex"
index <- 1:3
tdownloader(url,index)
\end{lstlisting}

\noindent \textbf{索引类型 3:含字符+统一格式数字文件}
\begin{lstlisting}
source("downloadhelper.R")
url <- "http://www1.umn.edu/statsoft/doc/statnotes/stat01.txt"
index <- 1:8
tdownloader(url,index)
\end{lstlisting}

\section{更新日志}

\begin{itemize}
    \item 2014-09-06:创立文件,内含 \lstinline{pkgtest\transform\downloader} 函数
    \item 2014-09-07:新增 tdownloader 函数,减少 4 个参数(皈颖协助)。
    \item 2014-09-15:增加中文 PDF 说明。
\end{itemize}
\newpage
\section{源码}
\begin{lstlisting}
# Copy Right by Ethan Deng (http://ddswhu.com)
# Email: ddswhu@gmail.com
# Last Modification: 2014-9-15
# deteck whether the downloader pkg has been installed
# if not, then install the pkg
# otherwise, return the information that the pkg is not installed
pkgtest <- function(x="downloader") {
    if (x %in% rownames(installed.packages()) == FALSE){
        install.packages(x)
    } else {print("The required packages have been installed")}
}

# transform the number index to special character
# index which has the same number of digits
# Example: 1:13 -> 01-09,10,11,12,13
transform <- function(index,ndigits=2){
    nullnum <- 10^{ndigits}
    cindex <- as.character(index + nullnum)
    dindex <- substr(cindex,2,ndigits+2)
    return(dindex)
}

downloader <- function(urlpre, index, transform=FALSE , ndigits = 2, filepre = "", filetype = "pdf") {
    # test whether the pkg is installed
    # load the pkg
    pkgtest("downloader")
    library(downloader)
    # If the transform is set to be TRUE and class of index
    # is not character, then the dindex should be transformed
    # using transform function defined
    if (transform == TRUE & !class(index) == "character"){
        dindex <- transform(index,ndigits)
    } else {
        dindex <- index
    }
    # sometimes, we have filepre such as "chap" or "chapter"
    # filetype can be "pdf","csv","xlsx","mp3"
    fileindex <- paste(filepre,dindex,".",filetype,sep="")

    # to create the full path of the files
    # urlpre indicates the url prefix
    urlindex <- paste(urlpre,fileindex,sep="")

    # binary files and text file list
    textfilelist <- c("csv","txt","R","tex","r","m")

    # mode is set to be "w" short for write(default)
    # if the file is text files, then it should be set
    # to be wb, short for write binary files
    if (filetype %in% textfilelist == FALSE){
        mode <- "wb"
    } else {
        mode <- "w"
    }

    # download process
    for (i in 1:length(index)) {
         download(url = urlindex[i], destfile = fileindex[i], mode = mode)
    }
}

tdownloader <- function(url,index){
    medfile <- strsplit(url,"\\/",fixed=FALSE)
    finalfile <- medfile[[1]][length(medfile[[1]])]
    urlpre <- substr(url,1,nchar(url)-nchar(finalfile)) # option 1
    filesplit <- strsplit(finalfile,"\\.",fixed=F)[[1]]
    filetype <- filesplit[2] # option 6
    fchar <- strsplit(filesplit[1],"[0-9]+",fixed=F)
    filepre <- ifelse(is.numeric(index), fchar[[1]], "")
     # option 5
    fnum <- strsplit(filesplit[1],"[a-z|A-Z]*",fixed=F)
    fnumb <- substr(filesplit[1],nchar(fchar[[1]])+1,nchar(fchar[[1]])+length(fnum[[1]])-1)
    transform <- !nchar(fnumb)==nchar(as.numeric(fnumb)) # option 3
    ndigits <- nchar(fnumb) # option 4
    downloader(urlpre,index,transform,ndigits,filepre,filetype)
}
\end{lstlisting}

\end{document}
